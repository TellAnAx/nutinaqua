Fishes as well as plants require a set of elements that have to be provided via the diet or nutrient solution and can be divided into macro- and micronutrients, according to their required quantities. However, they differ both in the required elements and their abundance. Table \ref{tab:essentials} provides an overview of the essential nutrients.

% Both fishes and plants
Both fishes and plants require carbon, oxygen, nitrogen, and sulphur as essential building bricks of the cells and for cellular respiration in large quantities. 

Further essential macronutrients for both are phosphorus, calcium, and magnesium. 
All organisms require phosphorus as essential building block for their \gls{dna and \gls{rna} and for the storage of energy in form of \gls{atp}. It is also the major constituent of phospholipids that form cell membranes. In fish, phosphorus is, in combination with calcium, also required for the development and maintenance of the skeletal system [@Lall2021]. Plants also require calcium for their growth. Biochemical processes such as cell elongation and division are affected by calcium levels [@Mengel2001].
Magnesium contributes to many biological processes in all organisms, from protein synthesis, over cell replication to the energy metabolism. In fish, it also plays a key role in signal transmission via the nervous system, among others [@Lall2021]. Magnesium is the central atom of chlorophyll, thus being crucial for the functioning of photosynthesis [@Mengel2001].

Essential micronutrients required by both fish and plants are iron, copper, manganese, and zinc. 
Iron and copper are important constituents of many metalloproteins and metalloenzymes in both fish and plants, such as heme proteins, cytochromes, among others [@Marschner2012; @Lall2021]. After iron, zinc is the second most abundant micronutrient in living organisms. Being part of metalloenzymes, it does not take part in redox reactions. Instead, it stabilises the enzyme structure [@Lall2021; @Marschner2013]. Manganese is an important co-factor for the activation of enzymes. It is also part of some metalloenzymes [@Marschner2013. In vertebrates, it is also involved in bone mineralisation [@Lall2021]. 

% Only fishes
A trace elements with proven relevance for fish but without physiological function in plants is iodine. Fish require iodine as constituent of the thyroid hormones that play a crucial role in the regulation of cell growth and functioning [@Lall2021].

% Only plants
Meanwhile, essential plant nutrients that are not essential for fish comprise boron, nickel, molybdenum, and chlorine [@Marschner2013].
Boron has been suggested to have several roles in metabolism, formation of cell wall structure, and substance transport. However, even though required in the largest amounts of all plant micronutrients, the physiological role of boron has not been fully understood [@Marschner2013]. Nickel is another micromineral for which essentiality has been shown but whose functions are not yet fully explored. As transition metal, it is also a constituent of metalloproteins and -enzymes [@Marschner2013]. Molybdenum is mostly involved in the nitrogen metabolism, either as constituent or co-factor of metalloenzymes [@Marschner2013].
Chlorine is an indispensable element for the maintenance of the ion balance and the stabilisation of the membrane potential. Even though considered a micronutrient due to it's low requirement, it is present in amounts that are comparable with macronutrients [@Marschner2013].