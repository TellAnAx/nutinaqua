\subsection{Aquafeeds}
As shown in table \ref{tab:essentials}, fish and plants differ in the minerals that are essential for them, but feed formulators do not controll for all nutrients essential for fish. Considerable variability could thus be expected for those nutrients that can





Back then, a large variability of mineral inclusion rates was noted and ascribed to differences in the composition of the utilised raw materials, but also differences in the composition of mineral premixes originating from different producers and contaminations of raw materials.







To date, however, no study has analysed a wider range of aquafeeds with regard to their mineral composition.




This is acknowledged by the availability of specific feeds for aquaculture in open (ponds or cages) and closed (\gls{ras}) systems to reduce the nutrient load of effluents or improve faeces stability and thus removability by mechanical filtration.

Aquaponics feeds, meanwhile, contain high proportions of phosphorus and potassium. The formulation of a specific aquaponics feed that meets the nutritional requirements of both fish and plants by leading to higher excretion of otherwise deficient plant nutrients is generally thought to be the most suitable approach for the development of an “off-the-shelf” nutrient delivery system that does not require additional management effort by aquaponics farmers [@Lennard2017; @Eck2019].


% Inclusion levels of nutrients

 This might be due to the fact that nutritional requirements of some of the mentioned substances have not yet been established for aquatic animals. Thus, there is currently no incentive to include those nutrients into feed formulation or to monitor their inclusion rates, making considerable variation between feeds likely.

In addition, current European regulations affecting labelling of compound feeds for aquaculture (defined by Regulation EC No. 767/2009 of the European Parliament) list only crude protein, crude fibre, crude oils and fats, crude ash, calcium, sodium and phosphorus as mandatory values to be stated on labels of commercial feeds. There is thus no obligation to analyse metals other than the stated ones. As a result, the most recent overview publication looking into metal inclusion rates in aquafeeds dates back 40 years [@Tacon1983].





- N data

After exclusion of specialty feeds, the range in which phosphorus, potassium, and calcium can be found in modern commercial fish feeds is comparably narrow, ranging from \SIrange{7}{18}{\gkg}, \SIrange{10}{13}{\gkg}, and \SIrange{12}{20}{\gkg}, respectively. These nutrients also show the lowest \gls{cv} (between \SIrange{20}{21}{\p}). The average inclusion rates did not significantly change in comparison with past values, being \SI{12(2.5)}{\gkg}, \SI{11(2.3)}{\gkg}, and \SI{18(3.7)}{\gkg} of phosphorus, calcium, and potassium in nowadays' feeds, respectively, while \SI{14.9(0.02)}{\gkg}, \SI{9.6(1.0)}{\gkg}, and \SI{20.1(2.7)}{\gkg} were reported before [@Tacon1983]. Magnesium inclusion rates are slightly more variable, ranging from \SIrange{1}{3}{\gkg} with an average of \SI{2(0.8)}{\gkg} (\SI{2.1(0.1)}{\gkg} in the past) and a \gls{cv} of \SI{36}{\p}.

In comparison with past values, the highest increase in their inclusion rates was found for the microminerals.

The iron content of aquafeeds increased by `r (1-183.5/209)*100`\si{\p}, with an average of \SI{209(237.1)}{\mgkg} in comparison with \SI{183.5(24.7)}{\mgkg} in the 90ies. Iron is also the nutrient with the highest variability in aquafeeds, indicated by the \gls{cv} of \SI{114}{\p} and a range from \SIrange{40}{544}{\mgkg}.

Nowadays, aquafeeds are richer in copper than in the past. The average inclusion rate increased by `r (1-15/20)*100`\si{\p}, from \SI{15(1.4)}{\mgkg} to \SI{20(18.5)}{\mgkg}. In the case of zinc, inclusion rates increased by `r (1-90/207)*100`\si{\p}, from \SI{90(7)}{\mgkg} to \SI{207(126.6)}{\mgkg}. The inclusion rates of both copper and zinc were again highly variable, with \gls{cv} of \SI{93}{\p} and \SI{61}{\p}, and ranges from \SIrange{5}{46}{\mgkg} and \SIrange{90}{384}{\mgkg}, respectively.



% Aquaponics feeds

There are some commercial feeds available that are marketed as specifically formulated for aquaponic systems (n = 3). Though, an in-depth characterisation of these feeds was not done until now to the best of the author's knowledge.



However, even though available data is limited, it can be seen that the phosphorus content of these aquafeeds is highly variable (\gls{cv} = \SI{76}{\p}). The average phosphorus content of \SI{22(17)}{\gkg}, meanwhile, is not significantly different from the unspecific diets (t-test: p = 0.4). Information with regards to potassium and calcium inclusion rates were reported only for one feed. Here, potassium and calcium inclusion rates were `r (1-20/11)*100`\si{\p} higher and `r (1-15/18)*100`\si{\p} lower than the average of unspecific diets, respectively. Other data with regards to these feeds is not available to the best of the author's knowledge.




% Experimental feeds

In the recent past, experiments were conducted to closer examine the manipulability of substance excretion patterns by fish in relation to alterations in the raw material composition of provided aquafeeds.



The overall nutrient composition of the experimental diets is comparable with commercially available aquafeeds, with no significant differences in average nutrient inclusion levels. Potassium (t-test: p = 0.3), phosphorus (t-test: p = 0.1), calcium (t-test: p = 0.4), and magnesium (t-test: p = 0.7) had average inclusion rates of \SI{9(2.8)}{\gkg}, \SI{14(4.9)}{\gkg}, \SI{15(8.4)}{\gkg}, and \SI{2(0.7)}{\gkg}, respectively.
Also, no differences were found for iron (t-test: p = 0.4), copper (t-test: p = 0.7), and zinc (t-test: p = 0.2). The average inclusion rates were \SI{348(29)}{\mgkg}, \SI{16(2.9)}{\mgkg}, and \SI{93(35.8)}{\mgkg}, respectively.



% Sources and extent of variability
The variability found in the data can be traced back to three causative factors, that are 1) the nutritional requirements of the fish species for which the feed was formulated, 2) the current approach of feed formulation based on nutrient digestibility, and 3) variability of the raw material composition among batches.

[@Prabhu2016]

% Fish species
@Gebauer2023
@Shaw2022
@Shaw2022a

% Ingredients and feed formulation
Feed formulators need to react to fluctuations of market prices for raw materials and the natural variation in raw material composition.


@Shaw2022
@Shaw2022a
@Gebauer2023
@Seawright1998

IAFFD database - check variability









Following: [@Lall2003]
K (water, feed)
Na (water)
Cl (water)
Following: [@Lall2021]
Ca (gills)
I (80% water rainbow trout)
Cr (gills)
Co (gills and feed)
Cu (feed; gills can contribute more than half of demand)
P (feed)
Mg (feed)
Fe (feed)
Se (feed)
Zn (feed)
Mn (unknown)



% Livestock as causative factor for variance
When discussing aquafeeds as nutrient input route, digestibility and retention of nutrients by the livestock must be considered.

The indigestible mass fraction of ingested feed, together with uneaten feed, make up the solid wastes. The fraction of feed that is digestible is meanwhile utilised to sustain the animals’ basal metabolism, somatic growth and reproductive activity. Metabolic end products are then excreted via the gills and the urinary system in dissolved form [@Hardy2003; @Evans2005]. Only the dissolved fraction of the nutrients is immediately available for plant uptake via the root system.

The availability of \gls{adc} and nutrient retention data for individual nutrients depends on whether the nutrient is essential and has to be considered for feed formulation.


% overview of ADCs determined for plant nutrients


Nitrogen that is, for the most part, present as \gls{cp} in aquafeeds, is generally well-digestible, with an \gls{adc} usually being above \SI{70}{\p} and on average approximately \SI{90}{\p} [@Guillaume2001; @IAFFD2021]. The excretion of N as end product of the protein and amino acid catabolism takes place in form of ammonia (\ce{NH3}) and, to a small extent, urea. The predominant excretory site are the gills, followed by renal excretion [@Dabrowski2003].

A less digestible nutrient is phosphorus, with \gls{adc} ranging from \SI{70}{\p} to only \SI{40}{\p} and a resulting excretion of \SI{30}{\p} to \SI{60}{\p} of the supply [@Lall2003; @Sugiura2018]. Especially plant ingredients in aquafeeds can cause low \gls{adc} if they are rich in phytic acid. Phytic acid is poorly digestible and can furthermore reduce the digestibility of minerals in the feed. This might also explain contradictory information in literature with reported renal excretion rates of \SI{90}{\p} of the total excreta [@Lall2003] in contrast to estimates of \SI{28}{\p} of excretion taking place in dissolved form and \SI{30}{\p} to \SI{64}{\p} excreted as particulate phosphorus [@Dabrowski2003].

Studies about \gls{adc} for the remaining essential plant nutrients are scarce. Variability of \gls{adc} among different feed ingredients was shown in Atlantic salmon (\emph{Salmo salar}) for Ca, Mg, Fe, Mn and Zn, with \gls{adc} ranging between \SIrange{30}{50}{\p} [@Sugiura1998]. Excretion of the earth alkaline metals calcium and magnesium primarily occurs in dissolved form via the gills and urine [@Oikari1985; @Lall2003]. Manganese, in contrast, is mostly excreted in solid form with feces, while renal excretion was found to be negligible [@Lall2003]. Copper is predominantly excreted via the bile [@Bury2003]. Excess dietary copper is not taken up but excreted with feces. Copper inclusion rates in aquafeeds are thus reduced to minimise its release into the environment [@Lall2003]. Excretion of Zinc mostly takes place renally and via the gills [@Lall2003].


