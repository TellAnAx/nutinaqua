\begin{table}
\centering
  \begin{threeparttable}
  \caption{Average percentage contribution of selected alkalinity supplements to the nutrient budget at different levels of makeup water per \si{\kg} of feed input supplied to the system. The required amount of alkalinity supplements was calculated assuming a feed contains \SI{35}{\p} \gls{cp} of which \SI{75}{\p} are excreted in form of \ce{NH4+} (Timmons et al., 2010).}
  \label{tab:alkalinitycont}
    \begin{tabularx}{\textwidth}{XXccc}

    \toprule
  
    \multirow{2}{*}{Substance}
    & \multirow{2}{*}{Makeup water}
    & \multicolumn{3}{c}{Contribution [\si{\p}]}
    \\
    
    \addlinespace
    \cline{3-5}
    \addlinespace
  
    &
    & Alkalinity suppl.
    & Feed
    & Water\tnote{+}
    \\

    \midrule

    K
    & \multirow{3}{*}{\SI{10}{\L\per\kg}}
    & 91.6
    & 8.3
    & 0.1
    \\

    Ca
    &
    & 75.7 
    & 23.0
    & 1.3
    \\

    Mg
    &
    & 94.4
    & 4.8
    & 0.8
    \\

    \addlinespace

    K
    & \multirow{3}{*}{\SI{100}{\L\per\kg}}
    & 91.1
    & 8.3
    & 0.7
    \\

    Ca
    & 
    & 67.6
    & 20.6
    & 11.8
    \\

    Mg
    & 
    & 88.4
    & 4.5
    & 7.1
    \\

    \addlinespace
    
    K
    & \multirow{3}{*}{\SI{1000}{\L\per\kg}}
    & 85.8
    & 7.8
    & 6.4
    \\

    Ca
    &
    & 32.9
    & 10.0
    & 57.1
    \\

    Mg
    &
    & 54.2
    & 2.8
    & 43.0
    \\

  \bottomrule
  \end{tabularx}
    \begin{tablenotes}
      \item[+] assuming the municipal tap as sole source of water
    \end{tablenotes}
  \end{threeparttable}
\end{table}
