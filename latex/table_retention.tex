\begin{table}
\centering
  \begin{threeparttable}
  
  \caption{Retention of nutrients by some fish species.}
  \label{tab:retention}
  
    \begin{tabularx}{\textwidth}{Xlccc}
    
    \toprule
    
    \multirow{2}{*}{Substance}
    & \multicolumn{3}{c}{Literature}
    & \multirow{2}{*}{Total loss}
    \\
    
    \cmidrule(lr){2-4}
    
    
    & Species
    & Digestibility
    & Retention
    & 
    \\
    
    \midrule
    
    N
    & Common carp
    & 
    & \SI{15}{\p}[@Fry2018]
    &
    \\
    
    
    & Channel catfish
    &
    & \SI{22}{\p}[@Fry2018] 
    &
    \\
    
    
    & Rainbow trout
    &
    & \SI{42.5}{\p}[@Morales2018]
    &
    \\
    
    
    & Tilapia
    &
    & \SI{18}{\p}[@Fry2018]; \SI{48}{\p}\tnote{†}[@Roy2022]
    &
    \\
    
    \addlinespace
    
    P
    & Rainbow trout
    &
    & \SI{22}{\p}[@Morales2018]
    &
    \\
    
    
    & Tilapia
    &
    & \SI{26}{\p}[@Roy2022]
    & 
    \\
    
    \addlinespace
    
    Ca
    & Rainbow trout
    &
    & \SI{64.7}{\p}[@Morales2018]
    &
    \\
    
    \addlinespace
    
    Mg
    & Tilapia
    &
    & \SI{15}{\p}[@Roy2022]
    &
    \\
    
    \addlinespace
    
    Fe
    & Tilapia
    &
    & \SI{24}{\p}[@Roy2022]
    &
    \\
    
    \addlinespace
    
    Mn
    & Tilapia
    &
    & \SI{31}{\p}[@Roy2022]
    &
    \\
    
    \addlinespace
    
    Cu
    & Tilapia
    &
    & \SI{40}{\p}[@Roy2022]
    &
    \\
    
    \addlinespace
    
    Zn
    & Tilapia
    &
    & \SI{34}{\p}[@Roy2022]
    &
    \\
    
    \bottomrule

    \end{tabularx}
    \begin{tablenotes}
      \item[†] assumed based on ash digestibility
    \end{tablenotes}
  \end{threeparttable}
\end{table}
