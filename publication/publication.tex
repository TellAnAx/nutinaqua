\documentclass[preprint, 3p,
authoryear]{elsarticle} %review=doublespace preprint=single 5p=2 column
%%% Begin My package additions %%%%%%%%%%%%%%%%%%%

\usepackage[hyphens]{url}

  \journal{An awesome journal} % Sets Journal name

\usepackage{graphicx}
%%%%%%%%%%%%%%%% end my additions to header

\usepackage[T1]{fontenc}
\usepackage{lmodern}
\usepackage{amssymb,amsmath}
% TODO: Currently lineno needs to be loaded after amsmath because of conflict
% https://github.com/latex-lineno/lineno/issues/5
\usepackage{lineno} % add
  \linenumbers % turns line numbering on
\usepackage{ifxetex,ifluatex}
\usepackage{fixltx2e} % provides \textsubscript
% use upquote if available, for straight quotes in verbatim environments
\IfFileExists{upquote.sty}{\usepackage{upquote}}{}
\ifnum 0\ifxetex 1\fi\ifluatex 1\fi=0 % if pdftex
  \usepackage[utf8]{inputenc}
\else % if luatex or xelatex
  \usepackage{fontspec}
  \ifxetex
    \usepackage{xltxtra,xunicode}
  \fi
  \defaultfontfeatures{Mapping=tex-text,Scale=MatchLowercase}
  \newcommand{\euro}{€}
\fi
% use microtype if available
\IfFileExists{microtype.sty}{\usepackage{microtype}}{}

\ifxetex
  \usepackage[setpagesize=false, % page size defined by xetex
              unicode=false, % unicode breaks when used with xetex
              xetex]{hyperref}
\else
  \usepackage[unicode=true]{hyperref}
\fi
\hypersetup{breaklinks=true,
            bookmarks=true,
            pdfauthor={},
            pdftitle={Unraveling the sources of nutrients and their contributions to the total input in aquaponic systems},
            colorlinks=false,
            urlcolor=blue,
            linkcolor=magenta,
            pdfborder={0 0 0}}

\setcounter{secnumdepth}{5}
% Pandoc toggle for numbering sections (defaults to be off)


% tightlist command for lists without linebreak
\providecommand{\tightlist}{%
  \setlength{\itemsep}{0pt}\setlength{\parskip}{0pt}}







\begin{document}


\begin{frontmatter}

  \title{Unraveling the sources of nutrients and their contributions to
the total input in aquaponic systems}
    \author[Some Institute of Technology]{Alice Anonymous%
  \corref{cor1}%
  \fnref{1}}
   \ead{alice@example.com} 
    \author[Another University]{Bob Security%
  %
  }
   \ead{bob@example.com} 
    \author[Another University]{Cat Memes%
  %
  \fnref{2}}
   \ead{cat@example.com} 
    \author[Some Institute of Technology]{Derek Zoolander%
  %
  \fnref{2}}
   \ead{derek@example.com} 
      \affiliation[Some Institute of Technology]{
    organization={Big Wig University},addressline={1 main
street},city={Gotham},postcode={123456},state={State},country={United
States},}
    \affiliation[Another University]{
    organization={Department},addressline={A street
29},city={Manchester,},postcode={2054 NX},country={The Netherlands},}
    \cortext[cor1]{Corresponding author}
    \fntext[1]{This is the first author footnote.}
    \fntext[2]{Another author footnote.}
  
  \begin{abstract}
  This is the abstract.

  It consists of two paragraphs.
  \end{abstract}
    \begin{keyword}
    tap water \sep minerals \sep fish feed \sep 
    alkalinity supplements
  \end{keyword}
  
 \end{frontmatter}

\hypertarget{introduction}{%
\section{Introduction}\label{introduction}}

Nutrients enter aquaponic systems through the source water, alkalinity
supplements used for pH control, and the daily ration of feed for the
aquatic livestock, hereinafter denoted as aquafeed (Rakocy et al.~2006,
Eck et al.~2019, Robaina et al.~2019).

The importance of water as nutrient source is usually seen as negligible
(Schmautz et al.~2016). However, it was found that it can introduce
large amounts of some plant nutrients such as Ca, Mg and S (Delaide et
al.~2017). Though, water can vary in its composition, depending on its
origin. A glimpse of the extent of variability of terrestrial waters in
their chemical composition is provided in Figure 1.1. Weathering of
rock, ion exchange, redox reactions and the buildup of biomass are the
main processes that are affecting concentrations of the shown compounds
(Stumm 1981). The highest variability can be found in anionic compounds
such as nitrate (\{\mathrm{NO}\}\_3\^{}-) and sulfate
(\{\mathrm{SO}\}\_4\^{}\{2-\}) and the cationic alkaline earth metals
\{\mathrm{Ca}\}\^{}\{2+\} and \{\mathrm{Mg}\}\^{}\{2+\} that are
covering a concentration range of approximately two orders of magnitude.

The term alkalinity supplements summarizes several alkaline substances
that are used to maintain a stable pH in the water (Timmons 2010).
Nitrification decreases the pH over time and, consequently, also the
activity of nitrifyers (Ward et al.~2011). Thus, a stable pH has to be
maintained to ensure both high nitrification performance and animal
welfare. For this purpose, several Na-based substances such as sodium
hydrogen carbonate (baking soda, \{\mathrm{NaHCO}\}\_3) are commonly
used in aquaculture due to their high and rapid solubility at a
comparably cheap price. However, high sodium (Na) concentrations must be
avoided in aquaponic systems due to its phytotoxicity (Maathuis et
al.~2014). Na is therefore replaced with several other supplements based
on K or Ca that come with the benefit of providing an additional source
of nutrients besides increasing the pH.

Aquafeeds are considered to be the most important source of nutrients,
providing a large and continuous nutrient input into the system.
Formulating a specific aquaponics feed was thus thought to be the most
suitable approach to develop an ``off-the-shelf'' nutrient delivery
system (Lennard 2017, Eck et al.~2019). When discussing aquafeeds as
nutrient input route, it is important to consider pathways of diet
utilization, as shown in Figure 1.2. Uneaten feed and the indigestible
mass fraction of ingested feed make up the solid wastes. The digestible
mass fraction is meanwhile utilized to sustain the animals' basal
metabolism, somatic growth and reproductive activity. Metabolic end
products are then excreted via the gills and the urinary system in form
of dissolved matter (Hardy 2003, Evans et al.~2005). Only the dissolved
fraction of the nutrients is immediately available for plant uptake via
the root system.

The availability of digestibility, retentiton and excretion data for
individual nutrients, with the former expressed with apparent
digestibility coefficients (ADC), depends on whether the nutrient is
essential and has to be considered for feed formulation. N that is, for
the most part, present as crude protein (CP) in aquafeeds, is generally
well-digestible, with ADCs usually being above 70\% and on average
approximately 90\% (Guillaume et al.~2001, International Aquaculture
Feed Formulation Database (IAFFD) 2021). The excretion of N as end
product of the protein and amino acid catabolism takes place in form of
ammonia (\{\mathrm{NH}\}\_3) and, to a small extent, urea. The
predominant excretory site are the gills, followed by renal excretion
(Dabrowski \& Guderley 2003). A less digestible nutrient is P with ADC
ranging from 70\% to only 40\% and a resulting excretion of 30\% to 60\%
of the supply (Lall 2003, Sugiura 2018). Especially plant ingredients in
aquafeeds can cause low ADC if they are rich in phytic acid. Phytic acid
is poorly digestible and can furthermore reduce the digestibility of
minerals in the feed. This might also explain contradictory information
in literature with reported renal excretion rates of 90\% of the total
excreta (Lall 2003) in contrast to estimates of 28\% of excretion taking
place in dissolved form and 30\% to 64\% excreted as particulate P
(Dabrowski \& Guderley 2003). Studies about ADCs of the remaining plant
nutrients are scarce. Variability of ADC among different feed
ingredients was shown in Atlantic salmon (Salmo salar) for Ca, Mg, Fe,
Mn and Zn, with ADCs ranging between 30\% to 50\% (Sugiura et al.~1998).
Excretion of the earth alkaline metals Ca and Mg primarily occurs in
dissolved form via the gills and urine (Oikari \& Rankin 1985, Lall
2003). Mn, in contrast, is mostly excreted in solid form as feces, while
renal excretion was found to be negligible (Lall 2003). Cu is
predominantly excreted via the bile (Bury et al.~2003). Excess dietary
Cu is not taken up but excreted as feces. Cu inclusion rates in
aquafeeds are thus reduced to minimize its release into the environment
(Lall 2003). Excretion of Zn mostly takes place renally and via the
gills (Lall 2003). Given the above-stated information, an animal
nutritionist's approach to develop a tailor-made aquafeed for aquaponics
might follow the paradigm that a target concentration of nutrients, for
instance given by commonly used hydroponic nutrient solutions in
hydroponics such as Hoaglands solution (Resh 2016) is taken as template.
Inclusion rates of the respective nutrients might eventually be
back-calculated using nutrient digestibility and retention data. Within
this context, it is necessary to ensure those nutrients can be supplied
without facing the risk of over- or undersupply. Therefore, it is
necessary to have profound knowledge about the variability of nutrient
concentrations within their respective nutrient sources. Furthermore, it
needs to be considered that physico-chemical constraints might limit the
concentration of a substance in solution. Accordingly, previous studies
found that the concentrations of some nutrients were not responsive to
elevated inclusion rates in aquafeeds. It is thus worth to extend the
nutritionist's approach, illuminating the ``aquaponic dark room'' with
respect to the chemical fate and behavior of nutrients.

\hypertarget{objectives}{%
\subsection{Objectives}\label{objectives}}

This study reviews aquaponics studies in an attempt to exemplary
identify the average contribution of different nutrient sources and
their variability with respect to the total daily nutrient inputs.
Recommendations with regards to the potential formulation of tailored
aquafeeds and nutrient management in aquaponics shall eventually be
developed to enhance the overall performance and profitability of
aquaponic systems.

\hypertarget{methodology}{%
\section{Methodology}\label{methodology}}

\hypertarget{data-acquisition-and-wrangling}{%
\subsection{Data acquisition and
wrangling}\label{data-acquisition-and-wrangling}}

The data acquisition and processing steps are graphically summarised in
Figure 2.1. Literature was screened for studies that focussed on
nutrient dynamics in aquaponic systems, resulting in an initial dataset
(IDS) of 117 individual observations originating from 39 publications.
Selected literature comprised studies about permanently and on-demand
coupled freshwater aquaponics, sludge remineralisation studies and
hydroponic growth trials with water originating from aquaculture
systems. The cultivated species, location of the experimental site, pH,
volume of the system parts, daily water exchange rate, initial and final
bodyweight, stocking density, daily feeding rate, feed name, and the
concentrations of all essential plant nutrients in the source water,
feed, and system water were collected.

To calculate the proportion contribution of the source water, feed and
alkalinity supplements to the total daily nutrient input in the reviewed
studies, their average daily inputs per volume unit had to be
calculated. For this purpose, statistics describing the average
experimental system, system parameters and nutrient inputs were derived
from the IDS and merged with external data sources in a four-stage
process. The first step comprised the generation of system assumptions.
For this purpose, a filter was applied to the IDS, only including
studies that made use of an aquaculture unit, reported the total system
volume and the final bodyweight of the livestock at the end of the
experiment. The resulting filtered dataset (FDS1) held 39 observations
originating from 10 publications. FDS1 was used to calculate the average
system volume (\mathrm{V}\emph{\{tot\}; \mathrm{m}\^{}3), average
bodyweight (ABW; g), average number of livestock (AN), average livestock
density (AD; kg \mathrm{m}\^{}\{-3\}), and average water exchange rate
(AWE; \mathrm{m}\textsuperscript{3\{\mathrm{\ d}\}}\{-1\}). ABW was
defined as the mean bodyweight during the duration of experiment and was
calculated as the arithmetic mean of the initial and final bodyweight.
For the calculation of AD, AN and AWE, NA values were removed.
Eventually, the average biomass (ABM; kg) was computed by multiplication
of ABW with AN and unit conversion or, if only AD was given, by
multiplication of AD with \mathrm{V}}\{tot\}. Vice versa, in case AD was
not reported, it was calculated by dividing ABM by
\mathrm{V}\emph{\{tot\}. The average daily water exchange volume by
multiplication of \mathrm{V}}\{tot\} with AWE. The second step was
estimating the daily nutrient input via source water. For this purpose,
gathered location information in the FDS was used to search for water
analysis reports provided by the closest water treatment plant. Only
official reports provided by local authorities and water utilities in
the corresponding municipalities were used to ensure that analyses were
conducted according to internationally accepted laboratory standards.
Both literature data and collected water analysis reports originated
from a total of 12 countries, with 39 literature studies and 37 analysis
reports. The country distribution of the literature data compared with
gathered water analysis reports is shown in Figure 2.2. It was not
possible to obtain water analysis reports from Finland, Iran, Israel and
The Netherlands. Thus, reports from Austria, France, Portugal and Turkey
were included instead.

In water analysis, it is common practice to report the value of the
detection limit instead of the measured value in case that the analyte
concentration found was below the sensitivity of the instrument used for
measurement (Deutsches Institut für Normung e.V. 2008). Figure 2.3 shows
the proportion of data for all considered analytes in the water analysis
reports that were found to be below the detection limit. Retaining the
detection limits in the dataset leads to mean concentration estimates
that are too high. Therefore, concentration data was recalculated using
the cenmle() function from the R package NADA by maximum likelihood
estimation (MLE) (Helsel 2011). This ensured that the estimates for
nutrient concentrations in tap water were reliable. The output was
eventually used to calculate the two-sided 90\% confidence interval for
the arithmetic mean of the concentration of each nutrient to describe
the concentration range that will be found in 90\% of all cases where
tap water is used. Finally, multiplying the obtained concentrations with
AWE yielded the upper and lower limits of the 90\% confidence interval
of total daily individual plant nutrient inputs via source water.

In the third step, daily nutrient inputs via aquafeeds were estimated.
The IDS was filtered for studies reporting the name of the commercial
aquafeed used (if not selfmade), the CP inclusion rate and the feeding
rate. From the resulting FDS2, the average feeding rate (\mathrm{FR};
\%), CP inclusion rate (ACP) and the averages of all plant nutrients
were calculated. Due to lacking nutrient composition data, supplier
datasheets were used and amended by literature data in case of
utilization of commercial aquafeeds. Incomplete observations with
respect to experimental feeds were handled in the same way, merging data
from multiple publications if the same feed was used. NA values were
removed for the computation of averages. Eventually, the average daily
feed input (\mathrm{FI}; kg \mathrm{d}\^{}\{-1\}) was calculated by
multiplying ABM from the generated system assumptions with . By
multiplying \mathrm{FI} with the average plant nutrient inclusion rates
in the aquafeeds, the uncorrected total daily input of individual plant
nutrients via aquafeeds could be calculated. Obtained values were then
corrected by multiplication with apparent digestibility coefficients
(ADC), accounting for the digestibility of aquafeeds by fish. An ADC of
90\% was assumed for N whereas the ADC of all other nutrients was
assumed to be 50\% (Lall Fish Nutrition The Minerals, 2002). Finally, it
was assumed that 50\% of the digestible fraction is retained while the
remaining 50\% is excreted in dissolved form (Halver \& Hardy 2003). The
indigestible part of the feed is excreted as solid feces and was thus
assumed not to participate in chemical reactions in solution. Lastly, an
estimate for the daily input of alkalinity supplements was derived from
\mathrm{AFI}. First, the daily CP input was calculated by multiplying
\mathrm{AFI} with the percentage of \mathrm{CP} on dry matter basis in
the feed.

m\_\{\mathrm{CP}\}=\mathrm{FI}\cdot\mathrm{CP}  2.1 Eventually,
digestibility and retention were considered and \mathrm{CP} converted
into the mass of non-retained N by division through the Kjeldahl factor
f\_K=6.25.
m\left(\mathrm{N}\right)=\frac{m_{\mathrm{CP}}\cdot\mathrm{ADC} -r\cdot m_{\mathrm{CP}}\cdot\mathrm{ADC} }{f_K}  2.2
The activity of nitrifying bacteria in the biofilter follows the overall
simplified reaction equation (2.3) (Timmons 2010).
\{\mathrm{NH}\}\_4\^{}++2\{\mathrm{\ O}\}\_2\rightarrow{\mathrm{NO}}\_3\textsuperscript{-+2\{\mathrm{\ H}\}}++\{\mathrm{\ H}\}\_2\mathrm{O}  2.3
Per mole of ammonium (\{\mathrm{NH}\}\_4\^{}+) that is converted to
nitrate (\{\mathrm{NO}\}\_3\^{}-), two moles of protons (H+) are
released. Furthermore, it can be derived from equation (2.3) that
n\left(\{\mathrm{NH}\}\_4\^{}+\right)=n\left(N\right) and
2n\left(N\right)=n\left(\mathrm{H}\^{}+\right). Thus, the amount of
moles that have to be neutralized based on \mathrm{AFI} can be
calculated as
n\left(\mathrm{H}\^{}+\right)=2\cdot\frac{m\left(\mathrm{N}\right)}{M\left(\mathrm{N}\right)}  2.4
with the molar mass
M\left(\mathrm{N}\right)=14.007\{\mathrm{\ g\ mol}\}\^{}\{-1\}. The
chemical reaction with bases such as \{\mathrm{NaHCO}\}\_3, \mathrm{KOH}
or \{\mathrm{Ca(OH)}\}\_2 is eventually a simple neutralization reaction
of the pattern
\mathrm{H}\textsuperscript{++\{\mathrm{OH}\}}-\rightarrow\mathrm{H}\_2\mathrm{O}  2.5
The mass of Na, K or Ca can consequently be calculated by rearranging
equation (2.4) and using the correct stoichiometric conversion factor.
The daily cost per compound was finally computed by multiplication of
the necessary inputs of alkalinity supplement with market prices
gathered from www.alibaba.com.

\hypertarget{software}{%
\subsection{Software}\label{software}}

All calculations were conducted using R (v4.2.2) with RStudio (``Spotted
Wakerobin'' Release) as graphical user interface.

\hypertarget{results}{%
\section{Results}\label{results}}

\hypertarget{derived-rearing-assumptions}{%
\subsection{Derived rearing
assumptions}\label{derived-rearing-assumptions}}

Assumptions were derived from literature studies about aquaponics to
calculate average contributions of different nutrient sources, namely
the source water, aquafeeds and alkalinity supplements to total daily
nutrient inputs into the system. Analysis of the data presented in the
reviewed aquaponics studies revealed that the average total water volume
of the aquaculture systems was 4.53 \mathrm{m}\^{}3 with an average
volume of the rearing compartment of 2.81 \mathrm{m}\^{}3. An average
bodyweight, calculated as the average of the initial and final
bodyweight, of 311.18 \mathrm{g} and an average biomass of 40.98
\mathrm{kg} were found, resulting in an average stocking density of 7.69
\{\mathrm{kg\ m}\}\^{}\{-3\}. The average feeding rate was 2\%, which
yielded a daily feed input of 0.67 \{\mathrm{kg\ d}\}\^{}\{-1\} with an
average CP content of 41\% on dry matter basis. This gives a daily N
input of 43.952 \{\mathrm{g\ d}\}\^{}\{-1\}. Based on the average water
exchange rate of 3\%, a daily freshwater input of 0.1
\mathrm{m}\textsuperscript{3\{\mathrm{\ d}\}}\{-1\} was computed.

\hypertarget{water-feed-and-alkalinity-supplements}{%
\subsection{Water, feed, and alkalinity
supplements}\label{water-feed-and-alkalinity-supplements}}

The average nutrient composition of both aquafeeds and source water and
the lower and upper limits of the 90\% confidence interval for source
water are shown in Table 3.1. Figure 3.2 provides an additional
graphical presentation of the findings with emphasis on the variability
of nutrient contributions by source water. The highest daily nutrient
inputs with respect to the macronutrients N and P in the reviewed
studies likely originated from aquafeeds, being on average 99.2\% and
99.3\%, respectively. The overall amounts of these two nutrients
contributed by source water are negligible. Aquafeeds can also be
assumed being the main source of the micronutrients Fe, Mn, Cu, Zn, and
Mo, with an average contribution of 96.2\%, 97.2\%, 79.8\%, 93.9\% and
76.3\%, respectively. Among these nutrients, Cu and Mo showed the
highest variability with respect to the contribution of source water
between locations (Cu 90\% CI: 8\%, 29.7\%; Mo 90\% CI: 14.7\%, 29.3\%),
with up to almost a third of the daily inputs possibly entering the
system via the water. Meanwhile, source water likely had a comparably
high contribution to Mg and S. With 76.5\% for Mg (90\% CI: 72.1\%,
79.7\%) and 68.8\% for S (90\% CI: 58.9\%, 74.8\%), an average delivery
of more than half of the daily input was found. Also, a considerable
contribution to daily B (63.1\%; 90\% CI: 48.5\%, 48.5\%) and Ni
(56.2\%; 90\% CI: 33.3\%, 67.4\%) inputs probably originated from the
source water.

Inputs of the remaining two nutrients, K and Ca, were found to be
heavily depending on the alkalinity supplement used. In absence of the
alkalinity supplement, aquafeeds were found to have the greatest impact
on K with a contribution of 81.16\%, while a higher proportion of Ca
(69.62\%; 90\% CI: 65.51\%, 72.86\%) would enter the system on average
via the water, compared with an average of30.38\% of feed contribution.
The difference is more pronounced with respect to Na: 82.25\% (90\% CI:
73.11\%, 86.76\%) of the total input was calculated to originate from
water while only 17.75\% were found to enter the system via daily
feeding. However, the calculated quantities of alkalinity supplements
necessary to maintain a constant pH were found to dominate each input
scenario, resulting in a contribution of more than 80\% of the total
input of K, Ca, or Na in any case. Table 3.2 summarizes some
supplements, their properties and prices. Using one of these substances
would result in the supplement contributing 98.7\%, 86.5\% or 96.3\% of
the total K, Ca or Na input, respectively. Regarding the costs for
alkalinity supplements, Na based substances are the cheapest, with
prices of between 0.01 and 0.03 EUR d-1, followed by Ca based
supplements ranging in the same price class. The most expensive
supplements to be used are those based on K. These substances would
cause costs between 0.10 and 0.14 EUR d-1, thus being on average 6 times
more expensive than Ca or Na based substances.

\hypertarget{discussion}{%
\section{Discussion}\label{discussion}}

The contribution of different nutrient sources to the total daily
nutrient inputs into an aquaponic system were compared based on
assumptions derived from literature. It could overall be confirmed by
the current input scenario that aquafeeds serve as major nutrient input
route for N, P, K, Fe, Mn, Cu, Zn, and Mo after consideration of
digestibility and nutrient retention by fish. These results confirm
prior findings (Delaide et al.~2017). Considering variability among
locations, it could furthermore be confirmed that Ca, Mg, S, B and Na
originate from source water. However, the variability found in the case
of Ca \left(min\left(\mathrm{Ca}\right)=70\%\cdot m a
x\left(\mathrm{Ca}\right)\right) does not reflect the usual
concentration range of approximately two orders of magnitude found in
terrestrial waters (Stumm 1981). A study reviewing the Ca concentration
in tap waters from the USA and Canada found a range from 1 to 135
\{\mathrm{mg\ L}\}\^{}\{-1\} (Morr et al.~2006). Considering that the
obtained results were obtained making use of assumptions, these initial
assumptions need to be discussed. The species distribution with strong
emphasis on Tilapia (Oreochromis spp.) and African catfish (Clarias
gariepinus) matches with survey results obtained in Europe (Villarroel
et al.~2016), even though approximately half of the studies included in
the dataset were conducted in the USA. However, surveys conducted in the
USA also revealed that the species mostly cultivated in aquaponic
systems are Tilapia (Love et al.~2015, Pattillo et al.~2022). On the
other hand, the average stocking density of 7.69 kg m-3 that was used
for the nutrient contribution calculations can neither be considered
representative for the intensive cultivation of Tilapia nor for African
catfish. While densities between 10 and 50 kg m-3 are reported as
acceptable for cultivation in RAS systems (El-Sayed 2019), densities of
120 kg m-3 were reported to be tolerated by the latter species without
any adverse effects on fish health (Nieuwegiessen et al.~2009). Even
though salmonids such as Rainbow trout (Oncorhynchus mykiss) are seldom
reared in aquaponic systems, they can tolerate stocking densities up to
137 kg m-3. Though, normal stocking densities range between 10 and 20 kg
m-3. The assumptions are rather on the lower end for salmonids (Pennell
\& McLean 1996). Common carp (Cyprinus carpio) is irrelevant for
aquaponic systems due to its low market price. Though, it was found that
fish welfare can be guaranteed at a stocking density of around 28 kg
m-3, which is \approx4 times higher than the assumed density (Ruane et
al.~2002). Consequently, it can be expected that the contribution of
aquafeeds to the total daily nutrient input under intensive rearing
conditions for the mentioned fish species is considerably higher than
assumed in this study. Though, the presented nutrient contribution
scenario might be representative for the cultivation of Pikeperch due to
the comparably low stocking densities (\textasciitilde5 kg m-3)
recommended for its cultivation (Kestemont et al.~2015). The average FR
of 2\% is in line with recommendations for the husbandry of Tilapia,
Channel catfish (Ictalurus punctatus) and Common carp (Lovell 2003).
Another aspect with respect to the contribution of feeds is that the
feed composition is, in adaptation to the species to be reared,
variable. For instance, crude protein inclusion rates in commercial
aquafeeds for the mentioned species range from 32\% for O. spp. (Wilson
2003, El-Sayed 2019) to 50\% for Pikeperch (Geay \& Kestemont 2015).
Further variation in the nutrient composition is introduced by feed
manufacturers due to changes in its raw material composition made
because of fluctuations in feed ingredient availability and market
prices. While changes in the mass concentrations of crucial nutrients
such as the crude protein (CP) inclusion rate are acceptable only within
tight limits, this might not be the case for other nutrients that are
disregarded in the formulation process. Overall, data about the
inclusion rates of nutrients relevant for plants in commercial aquafeeds
can be considered scarce because these compounds are either not
essential for aquatic animals or can be taken up from water via the
gills in sufficient quantities. These nutrients are thus usually not
considered during feed formulation (Lall 2003). However, a comprehensive
review about the range of plant nutrient mass concentrations in
commercial aquafeeds has not yet been conducted to the best of the
authors knowledge. The mean water exchange rate of 3\% can be considered
at the lower end of commonly reported water exchange rates for RAS and
within the range recommended for aquaponic systems (Timmons 2010). Using
analysis reports from municipal water suppliers can be considered as
valid approach with respect to the acquisition of nutrient concentration
data in source water as it was found by a prior survey that educators,
including research institutions, usually use tap water (Love et
al.~2015). It could be further argued that, due to its low water demand,
aquaponics is generally seen as food production system suitable for
urban areas or arid regions (Kloas et al.~2015, Joyce et al.~2019).
Establishing an aquaponic system in these regions comes with limited
access to water sources such as well water or rivers and lakes.
Rainwater, on the other hand, would require large storage capacities.
Tap water is thus assumed to be the most important water source.
Recalculation of nutrient concentrations as response to high proportions
of censored data where only the limit of detection is reported is a
rather uncommon approach. However, removing the affected observations
would have resulted in a loss of a large proportion of data and, in
addition, would have led to an average value that is higher than the
estimates obtained by recalculation because the remaining data consisted
only of concentration values that are high enough to be determined. A
similar result would have arisen from using the reported detection
limits for the calculation of nutrient concentration means. The results
of the nutrient concentration recalculation via MLE procedure can thus
be considered closer to the true concentration means than the outcome of
the other mentioned approaches. The daily quantity of alkalinity
supplements to be added might be overestimated. In biofilters, both
nitrification and denitrification processes usually occur due to
partially anoxic conditions, for instance within biofilms.
Denitrification yields alkalinity (Timmons 2010). However, the
contribution of alkalinity supplements will likely still exceed 50\% of
the total daily nutrient inputs. Furthermore, the calculations assumed
that the supplements would have a purity of 100\%. In practice a lower
purity, corresponding to a food grade certification, can be expected.

\hypertarget{implications-for-the-formulation-of-tailored-aquaponics-feeds}{%
\subsection{Implications for the formulation of tailored aquaponics
feeds}\label{implications-for-the-formulation-of-tailored-aquaponics-feeds}}

The development of a dedicated aquaponics feed is seen as potentially
user-friendly and rapid way of adding nutrients to the system, resulting
in close-to optimum concentrations for the plants. However, the results
of this study imply several limitations for feed formulation. Firstly,
considering potential precipitation, target nutrients for an increased
delivery could include N, K, Mg, S, B, and Zn. This suggestion is partly
confirmed by prior empirical studies. One of those found K, Mg, and Zn
to accumulate out of the mentioned nutrients, with additional
accumulation of P, Mn and Zn (Seawright et al.~1998). Another study
reported N, K, and Mg accumulation alongside P and Cu (Shaw et
al.~2022). The focus on K and Mg can thus be considered ``safe'',
considering the generally high solubility of both compounds (Lide 2007)
and reported deficiencies for both in aquaponics (Lunda et al.~2019).
Natural feed ingredients rich in K are soy lecithin (13.0\% K on dry
matter basis), vinasse (3.6\% K on dry matter basis) and some protein
feedstuffs such as poultry by-product meal (3.5\% K on dry matter basis)
and hydrolyzed fish solubles (2.4\%-2.9\% K on dry matter basis). Mg can
be introduced to aquafeeds mostly via the use of seaweed (3.9\% Mg on
dry matter basis) (International Aquaculture Feed Formulation Database
(IAFFD) 2021). The drawback of the use of some of the above-stated feed
ingredients is that the target nutrient is usually accompanied by
comparably high amounts of Na. Even though this study shows that
aquafeeds were not the main input route of Na in the reviewed
literature, it is nevertheless reasonable to attempt to decrease Na
inclusion rates during feed formulation because the initial assumptions
leading to the nutrient input scenarios are not representative for
commercial systems, as described above. This could be done for instance
by replacement of feed ingredients with high Na inclusion rates such as
fish meal originating from marine fish by other protein feedstuffs.
Another possibility is given by using food-grade salts of the desired
nutrients. However, considering the nutrient input via feed (1.1\%) in
comparison with that of alkalinity supplements containing K (98.7\%) in
this study, it depends on target concentrations and costs whether it is
reasonable or not to increase the inclusion rate of K in a tailored
aquafeed versus using the respective alkalinity supplements.
Furthermore, it remains questionable whether it is more meaningful to
include, for instance, salts into the feed or to add them directly into
the system. Inclusion rates of nutrients that are confirmed to be
affected by precipitation (Ca, Fe) are, in turn, recommended to be
decreased in case the nutrient is not essential for the livestock and a
change in feed formulation economically viable. Reducing the amount of
precipitated nutrients increases the overall nutrient use efficiency.
This also accounts for P, Mn, Cu, and Zn. No clear recommendation can be
given in terms of inclusion rates in aquafeeds tailored for aquaponics.
More research is needed to gain a better understanding of the behavior
of these compounds under aquaculture and aquaponic conditions.

\hypertarget{implications-for-the-management-of-aquaponic-systems}{%
\subsection{Implications for the management of aquaponic
systems}\label{implications-for-the-management-of-aquaponic-systems}}

The findings of this study can be generalised in terms of
recommendations for nutrient management and system design in aquaponic
systems. The idea behind the design of on-demand coupled aquaponic
systems as alternative to permanently coupled systems was to be able to
maintain optimum conditions for both livestock and plants with respect
to pH. Though, increasing the pH induces new problems by increasing the
concentration of \{\mathrm{PO}\}\_4\^{}\{3-\} and \{\mathrm{OH}\}\^{}-,
thus lowering the saturation concentration of other dissolved cations
such as \{\mathrm{Fe}\}\^{}\{3+\}. The consequence is uncontrolled
precipitation of affected nutrients within the aquaculture unit.
Avoiding this could be achieved by decreasing the pH, thus leading to a
shift in \{\mathrm{PO}\}\_4\^{}\{3-\} and \{\mathrm{OH}\}\^{}-
equilibrium concentrations, in permanently coupled systems. The impaired
nitrification performance, meanwhile, could be counteracted by creation
of microenvironments within the biofilter (Gieseke et al.~2006).
Meanwhile, nutrients might be more efficiently utilized in on-demand
coupled systems by means of ``nutrient harvesting'' from the aquaculture
unit: controlled precipitation of nutrients could be induced, for
instance in a settler, so that the sludge/precipitate mix can be used
for consequent remineralisation at lower pH. This could be done via
flocculation/coagulation (Ebeling et al.~2003). The best way to avoid
precipitation of nutrients is, however, their application as foliar
sprays. Good results were achieved with this technique (Roosta \&
Hamidpour 2013, Roosta 2014). A further discussion of the pros and cons
in terms of efficiency are out of scope of this study.

\hypertarget{future-research-needs}{%
\subsection{Future research needs}\label{future-research-needs}}

Aquaponics is an interdisciplinary field, combining both aquaculture and
horticulture but also aquatic chemistry. Implications arising from
chemistry are, however, not closer examined and data is largely lacking.
It is crucial for a thorough modeling of chemical equilibria that pH and
water temperature are reported. Furthermore, many studies were found to
report averaged nutrient concentrations. However, due to reaction
kinetics, time series data would help to gain a better understanding of
the nutrient dynamics within the system. Largely lacking are furthermore
studies assessing the effects of DOM as one of the key differences
between aquaponics and hydroponics, with only one in-silico study
published to the best of the authors knowledge (Silva Cerozi 2020). It
is long known that DOM is the main chelating agent in natural waters and
heavily affecting metals mobility (Stumm 1981). It was furthermore found
that the type and source of DOM dictates the microbial community
composition in different aquatic environments (van Hannen et al.~1999,
Landa et al.~2013). These results were recently expanded within a plant
context by the findings that the rhizosphere microbial community
composition might be the result of an active selection process exerted
by the plants (Lobanov et al.~2022). Lastly, it needs to be assessed
whether the total concentration of nutrients is actually of importance
for plants. Aquaponic plant growth studies usually compare plant
nutrient concentrations in aquaponics with commonly used hydroponic
nutrient solutions such as Hoagland's solution (Resh 2016). However, it
was reported that a reduction of macronutrient concentrations down to
50\% of the control level does not have adverse effects on growth, fruit
yield and fruit quality in tomato (Siddiqi et al.~1998). The
stoichiometry of nutrient solutions might thus play a more vital role
for plants, which needs to be closer examined.

\hypertarget{conclusion}{%
\subsection{Conclusion}\label{conclusion}}

Nutrient imbalances in aquaponics appear to be a result of chemical
constraints rather than imbalanced nutrient inputs. Aquaponic systems
differ from hydroponics in terms of pH and DOM concentrations.
Understanding the chemistry of aquaponics is thus of major importance to
improve nutrient cycling within the system. When trying to improve
system performance via formulation of a tailored aquafeed, reducing
inclusion rates of nutrients affected by precipitation (Ca, Fe) appears
to be more meaningful than increasing inclusion rates of well-soluble
nutrients (K, Mg). These can instead be supplied via alkalinity
supplements.

\hypertarget{references}{%
\section*{References}\label{references}}
\addcontentsline{toc}{section}{References}


\end{document}
